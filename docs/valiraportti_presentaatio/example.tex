\documentclass[first=dgreen,second=purple,logo=yellowexc]{aaltoslides}
%\documentclass{aaltoslides} % DEFAULT
%\documentclass[first=purple,second=lgreen,logo=redque,normaltitle,nofoot]{aaltoslides} % SOME OPTION EXAMPLES

\usepackage[latin9]{inputenc}
\usepackage[T1]{fontenc}
\usepackage{graphicx}
\usepackage{amssymb,amsmath}
\usepackage{url}
\usepackage{lastpage}

\title{Aalto Slides with \LaTeX:\\ \texttt{aaltoslides} Document Class for Beamer}

\author[K. J�rvinen]{Kimmo J�rvinen}
\institute[ICS]{Department of Information and Computer Science\\
Aalto University, School of Science and Technology\\kimmo.jarvinen@tkk.fi}

\aaltofootertext{Example Slides}{\today}{\insertframenumber/\inserttotalframenumber}

\date{Version 1.01, \today}

\begin{document}

%%%%%%%%%%%%%%%%%%%%%%%%%%%%%%%%%%%%%%%%%%%%%%%%%%%%%%%%%%%%%%%%%%%%%%%%%%%%%%%%%%%%%%%%%%%%%

\aaltotitleframe

%%%%%%%%%%%%%%%%%%%%%%%%%%%%%%%%%%%%%%%%%%%%%%%%%%%%%%%%%%%%%%%%%%%%%%%%%%%%%%%%%%%%%%%%%%%%%

\begin{frame}{Introduction}
\begin{itemize}
\item The purpose of \texttt{aaltoslides} is to allow us who use Beamer to easily produce slides that somewhat look like the official Aalto slides
\item \texttt{aaltoslides} has \alert{not} been approved by anybody responsible of the Aalto visual style
\item \texttt{aaltoslides} only resembles the official Aalto Powerpoint templates: font (Helvetica, not Nimbus Sans), sizes of the logos, footer bars, etc. may differ from the official ones
\item All comments are welcome! (kimmo.jarvinen@tkk.fi)
\end{itemize}
\end{frame}

%%%%%%%%%%%%%%%%%%%%%%%%%%%%%%%%%%%%%%%%%%%%%%%%%%%%%%%%%%%%%%%%%%%%%%%%%%%%%%%%%%%%%%%%%%%%%

\begin{frame}{What's New in Version 1.01}
\begin{itemize}
\item School of Science and Technology $\Rightarrow$ School of Science
\item The package now includes also EPS variants of the logos; i.e., \texttt{aaltoslides} now works also with plain \LaTeX
\end{itemize}
\end{frame}

%%%%%%%%%%%%%%%%%%%%%%%%%%%%%%%%%%%%%%%%%%%%%%%%%%%%%%%%%%%%%%%%%%%%%%%%%%%%%%%%%%%%%%%%%%%%%

\begin{frame}{Title Page}

\begin{itemize}
\item The largest difference to normal Beamer slides is that the title page is produced with a special command \texttt{\textbackslash aaltotitleframe}; this command should be used as it is. \alert{Don't put it in a frame environment!!!} See \texttt{example.tex}.
\item By default, \texttt{aaltoslides} uses a title page similar to the one in the Aalto Powerpoint templates
\item A more traditional looking title page can be selected with the option: \texttt{normaltitle}
\end{itemize}

\end{frame}

%%%%%%%%%%%%%%%%%%%%%%%%%%%%%%%%%%%%%%%%%%%%%%%%%%%%%%%%%%%%%%%%%%%%%%%%%%%%%%%%%%%%%%%%%%%%%

\begin{frame}{Aalto Colors}

\begin{itemize}

\item Colors are defined with options: \texttt{first=$\star$} and \texttt{second=$\star$} where $\star$ is one of the following:\\
{\color{aaltoyellow}yellow}, 
{\color{aaltored}red}, 
{\color{aaltoblue}blue}, 
{\color{aaltogray}gray}, 
{\color{aaltolgreen}lgreen}, 
{\color{aaltodgreen}dgreen}, 
{\color{aaltocyan}cyan}, 
{\color{aaltopurple}purple}, 
{\color{aaltomagenta}magenta}, or
{\color{aaltoorange}orange}

\item The first color is the primary color (titles, the footer bar, \ldots) and the second color is used in alerted texts and examples

\item Default colors are: {\color{aaltoblue}blue} and {\color{aaltored}red}

\item Colors can be used also with the command: \texttt{\textbackslash color\{aalto}$\star$\texttt{\}}. For example, 
\texttt{\{\textbackslash color\{aaltocyan\}some text\}} 
gives {\color{aaltocyan}some text}

\item The rules for choosing colors from the Aalto color circle are not checked (two adjacent colors should not be used)

\end{itemize}

\end{frame}

%%%%%%%%%%%%%%%%%%%%%%%%%%%%%%%%%%%%%%%%%%%%%%%%%%%%%%%%%%%%%%%%%%%%%%%%%%%%%%%%%%%%%%%%%%%%%

\begin{frame}{Logo}

\begin{itemize}
\item \texttt{aaltoslides} supports all variations of the logo:\\[0.2cm]
\includegraphics[width=3cm]{Aalto_EN_ScienceandTech_13_RGB_y1} \hspace{10pt}
\includegraphics[width=3cm]{Aalto_EN_ScienceandTech_13_RGB_y2} \hspace{10pt}
\includegraphics[width=3cm]{Aalto_EN_ScienceandTech_13_RGB_y3} \\[0.2cm]
\includegraphics[width=3cm]{Aalto_EN_ScienceandTech_13_RGB_r1} \hspace{10pt}
\includegraphics[width=3cm]{Aalto_EN_ScienceandTech_13_RGB_r2} \hspace{10pt}
\includegraphics[width=3cm]{Aalto_EN_ScienceandTech_13_RGB_r3} \\[0.2cm]
\includegraphics[width=3cm]{Aalto_EN_ScienceandTech_13_RGB_b1} \hspace{10pt}
\includegraphics[width=3cm]{Aalto_EN_ScienceandTech_13_RGB_b2} \hspace{10pt}
\includegraphics[width=3cm]{Aalto_EN_ScienceandTech_13_RGB_b3} \\
\item The logo is selected with the option: \texttt{logo=$\star\circ$} where $\star$ is {\color{aaltoyellow}yellow}, {\color{aaltored}red}, or {\color{aaltoblue}blue}, and $\circ$ is either exc, quo, or que for !, ", or ?, respectively; For example, \texttt{logo=yellowquo}
\item The title page uses a larger variation of the selected logo
\item By default, the logo is \texttt{logo=redexc}
\item All logos can be removed with the option: \texttt{nologo}
\end{itemize}

\end{frame}

%%%%%%%%%%%%%%%%%%%%%%%%%%%%%%%%%%%%%%%%%%%%%%%%%%%%%%%%%%%%%%%%%%%%%%%%%%%%%%%%%%%%%%%%%%%%%

\begin{frame}{Footer}

\begin{itemize}

\item By default, the slides have a footer with a logo on the left and an optional three row text\footnote{The first row is highlighted with black by default. To remove this, simply change the color of the first argument of  \texttt{\textbackslash aaltofootertext\{\}\{\}\{\}} with \texttt{\textbackslash color\{aaltogray\}}} on the right
\item The footer text is set up with the command: \texttt{\textbackslash aaltofootertext\{\}\{\}\{\}}
\item The footer can be removed with the option: \texttt{nofoot}

\end{itemize}

\end{frame}

%%%%%%%%%%%%%%%%%%%%%%%%%%%%%%%%%%%%%%%%%%%%%%%%%%%%%%%%%%%%%%%%%%%%%%%%%%%%%%%%%%%%%%%%%%%%%

\begin{frame}{Lengths}

\begin{itemize}
\item \texttt{aaltoslides} defines the following lengths: 
\texttt{\textbackslash aaltofooterplace}, 
\texttt{\textbackslash aaltofooterruleheight}, 
\texttt{\textbackslash aaltofooterrulewidth}, 
\texttt{\textbackslash aaltotitleboxplace},  
\texttt{\textbackslash aaltotitleboxheight}, 
\texttt{\textbackslash aaltotitleboxwidth}, 
\texttt{\textbackslash aaltotitlesep}, 
\texttt{\textbackslash aaltotitleentrysep}, 
\texttt{\textbackslash largelogoheight}, and
\texttt{\textbackslash smalllogoheight} 
\item The appearance of the slides can be changed by modifying the lengths with \texttt{\textbackslash setlength} or \texttt{\textbackslash addtolength}
\end{itemize}

\end{frame}

%%%%%%%%%%%%%%%%%%%%%%%%%%%%%%%%%%%%%%%%%%%%%%%%%%%%%%%%%%%%%%%%%%%%%%%%%%%%%%%%%%%%%%%%%%%%%


\begin{frame}{Some Examples}

\begin{itemize}
\item Normal text
\item \alert{Alerted text}
\end{itemize}

\begin{block}{Block 1}
Text
\end{block}

\begin{example}
Text
\end{example}

\end{frame}

%%%%%%%%%%%%%%%%%%%%%%%%%%%%%%%%%%%%%%%%%%%%%%%%%%%%%%%%%%%%%%%%%%%%%%%%%%%%%%%%%%%%%%%%%%%%%

\end{document}
